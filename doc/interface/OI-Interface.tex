% !TeX spellcheck = en_US
\documentclass{article}
\usepackage[affil-it]{authblk}

\usepackage{graphicx}
\usepackage[space]{grffile} % extends the file name processing of package graphics

%\usepackage{latexsym} % add some characters from the "lasy" fonts
\usepackage{amsfonts,amsmath,amssymb}

\usepackage[utf8]{inputenc}

\usepackage{OI-Interface}

\hypersetup{%
  pdftitle={},
  pdfauthor={},
  colorlinks=true,
  linkcolor=DarkCyan,
  citecolor=DarkGreen,
  filecolor=DarkCyan,
  menucolor=DarkOrange,
  urlcolor=DarkCyan,
  pdfborder={0 0 0}}

\usepackage{textcomp}
\usepackage{longtable}
\usepackage{multirow,booktabs}
\usepackage[authoryear,round]{natbib}

%==============================================================================
\newcommand*{\ROW}{} % predefined here, redefined later
\newcommand*{\ROWTITLE}{} % predefined here, redefined later

\newcommand{\oops}[1]{\DarkRed{#1}}
\newcommand{\KEYWORD}[1]{\texttt{#1}} % FITS keyword
\newcommand{\STRING}[1]{\texttt{'#1'}} % FITS string value
\newcommand{\VARIABLE}[1]{\texttt{\textsl{#1}}}
%==============================================================================


\begin{document}

\title{Interface to Image Reconstruction}

\author{Éric Thiébaut$^1$, John Young$^2$, Jonathan Léger$^1$,
  Michel Tallon$^1$, Gilles Duvert$^3$, Guillaume Mella$^3$, Laurent Bourges$^3$ \&
  Martin Vannier$^4$}

\affil{$^1$CRAL, $^2$University of Cambridge, $^3$JMMC/IPAG, $^4$Lagrange}


\date{\today}

\maketitle

%==============================================================================
\section{Introduction}

Many image reconstruction algorithms have been developed to process optical
interferometric data.  Using these algorithms may require substantial
expertise (for instance you may have to learn a specific programming
language). One of the objectives of the OPTICON JRA4 is to provide a common
graphical user interface (GUI) to drive such imaging algorithms in a
\emph{user friendly} way. For obvious reasons, we also want to avoid rewriting
the image reconstruction algorithms as much as possible.  A model of
interaction between these two pieces of software under these constraints has
to be developed. This document aims at specifying a common software interface
for image reconstruction algorithms.

On the image reconstruction side, this involves specifying the inputs
completely and precisely: OI-FITS file for the data, the initial
image, and other settings such as the regularization and its
hyper-parameters.\footnote{In a previous discussion, John Young
  suggested using a text file with a simple format to give the values
  of the input parameters. The format of this file must be simple and
  such that it can be easily edited by a human and easily read by a
  software program.  Here a different proposition is made which
  exploits the flexibility of the FITS format.} Given these inputs,
the image reconstruction algorithm should start the reconstruction
producing two kind of outputs:
\begin{description}
  \item[Intermediate outputs:]  These consist of iteration information
  printed on the standard output (for instance)  and captured by the GUI and
  also the current solution in the form of an image, the corresponding model
  of the measurements (\emph{e.g.}, the complex visibilities).  These outputs
  will be used by the GUI to display the current solution, to compare the
  actual data with the current model, \emph{etc.}

  \item[Final outputs:]  These are essentially the final image and consistent
  information about the pixel size, the orientation, \emph{etc.} but also
  about the input data (\emph{i.e.} to avoid a \emph{traceability issue}), the
  reconstruction method and its parameters.
\end{description}

A few related points have to be considered:
\begin{enumerate}
  \item The command line image reconstruction algorithms should (or not?)
  switch off their own display capabilities.

  \item There must be a way to interrupt the image reconstruction algorithms
  so as to make sure they do release all their resources (no zombie process,
  undetached shared memory, etc.).

  \item Restarting an image reconstruction (to continue the iterations
    or perhaps with a few settings changed) should simply be a matter
    of re-starting the algorithm with, as its input image, the last
    image produced.

  \item  In case of errors, we must adopt some kind of conventions to make
  sure that the GUI can properly account for the errors.

\item It would be helpful for the GUI to incorporate a facility
  whereby the user can pause the reconstruction, translate the image
  so that its centre-of-gravity is in the centre of the field, then
  continue the reconstruction.

\end{enumerate}

%The GUI will be developed at JMMC, presumably in Java as it is mostly portable
%and they already use this language for their developments. Furthermore, they
%have a mastering for the kind of software wrapping we want to have. For
%instance, having a GUI which drives a command line program is exploited by
%LITpro, the model fitting software of the JMMC.


To summarize, on entry:
\begin{itemize}
\item the data;
\item image parameters (dimensions, pixel size, \emph{etc.});
\item initial image;
\item regularization and hyper-parameters;
\end{itemize}
On output (for the final and intermediate results), \emph{in addition
  to the above list}:
\begin{itemize}
\item the current/final image;
\item the model of the data;
\item other relevant information;
\end{itemize}


To make the interfacing as simple as possible, it is proposed in this
document to use a single entry file containing all input parameters
and data and a single output file with all the results (either
intermediate or final).  The flexibility of the FITS file format is
exploited to store all these information and data.

All image reconstruction software must be able to run from the command line
and with two arguments: the names of the input and output files (possibly with
some flag to indicate that the software must be run in a specific way).   For
instance:
\begin{verbatim}
  command --batch input output
\end{verbatim}


%==============================================================================
\renewcommand{\ROW}[3]{\KEYWORD{#1} & #2 &#3 \\}
\renewcommand{\ROWTITLE}[1]{\multicolumn{3}{c}{\textbf{#1}} \\*
  Keyword & Type & Description \\*}

\begin{longtable}[c]{lcp{83mm}}
  %\centering
  \caption[FITS keywords]{FITS keywords used to specify the input
  parameters.  The \emph{Image Parameters} are stored in the HDU which
  contains the initial image. This HDU is specified by the value of the
  \KEYWORD{INIT\_IMG} keyword, which gives its \KEYWORD{HDUNAME}. All other
  parameters \oops{(except \KEYWORD{LAST\_IMG})} are in a binary table HDU
  with \KEYWORD{EXTNAME} of \STRING{IMAGE-OI~INPUT~PARAM}.
  \label{tab:input-params}}
  \endfirsthead
  \caption[]{(continued)}
  \endhead
%
  \hline
  \ROWTITLE{Data Selection}
  \hline
  \ROW{TARGET}{string}{Identifier of the target object to reconstruct}
  \ROW{WAVE\_MIN}{real}{Minimum wavelength to select (in meters)}
  \ROW{WAVE\_MAX}{real}{Maximum wavelength to select (in meters)}
  \ROW{USE\_VIS}{string}{Complex visibility data to consider if any$^\dagger$}*
  \ROW{USE\_VIS2}{logical}{Use squared visibility data if any}
  \ROW{USE\_T3}{string}{Bispectrum data to consider if any$^\dagger$}[1ex]
  \multicolumn{3}{c}{$^\dagger$ value can be: \STRING{NONE}, \STRING{ALL},
   \STRING{AMP} or \STRING{PHI}.}\\
  \hline
  \ROWTITLE{Algorithm Settings}
  \hline
  \ROW{INIT\_IMG}{string}{Identifier of the initial image}
  \ROW{MAXITER}{integer}{Maximum number of iterations to run}
  \ROW{RGL\_NAME}{string}{Name of the regularization method}
  \ROW{AUTO\_WGT}{logical}{Automatic regularization weight}
  \ROW{RGL\_WGT}{real}{Weight of the regularization}
%  \ROW{RGL\_ALPH}{real}{Parameter $\alpha$ of the regularization}
%  \ROW{RGL\_BETA}{real}{Parameter $\beta$ of the regularization}
%  \ROW{RGL\_GAMM}{real}{Parameter $\gamma$ of the regularization}
%  \ROW{RGL\_TAU}{real}{Parameter $\tau$ of the regularization}
  \ROW{RGL\_PRIO}{string}{Identifier of the HDU with the prior image}
  \ROW{FLUX}{real}{Assumed total flux (1 is the default)}
  \ROW{FLUXERR}{real}{Error bar for the total flux (0 means strict constraint)}  
  \hline
  \ROWTITLE{Image Parameters}
  \hline
  \ROW{HDUNAME}{string}{Unique name for the image within the FITS file}
  \ROW{NAXIS1}{integer}{First dimension of the image}
  \ROW{NAXIS2}{integer}{Second dimension of the image}
  \ROW{CTYPE1}{string}{\STRING{RA---TAN}}
  \ROW{CTYPE2}{string}{\STRING{DEC--TAN}}
  \ROW{CDELT{\slshape i}}{real}{Pixel increment along \texttt{\slshape i}-th
    dimension of the image (for \texttt{\slshape i} = 1 or 2)}
  \ROW{CUNIT{\slshape i}}{string}{Physical units for \KEYWORD{CDELT{\slshape i}}
    and \KEYWORD{CRVAL{\slshape i}}; defaults to \STRING{deg} if omitted}
  \ROW{CRPIX{\slshape i}}{real}{Index of reference pixel along \texttt{\slshape i}-th
    dimension (for \texttt{\slshape i} = 1 or 2); defaults to the geometric center
    of the field of view if omitted}
  \ROW{CRVAL{\slshape i}}{real}{Physical coordinate of reference pixel along
    \texttt{\slshape i}-th dimension (for \texttt{\slshape i} = 1 or 2) and
    relative to the center of the field of view; defaults to 0 if omitted}
  \hline
  \ROWTITLE{Algorithm Results}
  \hline
  \ROW{LAST\_IMG}{string}{Identifier of the final image \oops{(in \STRING{IMAGE-OI OUTPUT PARAM})}}
  \hline
\end{longtable}

\section{Input format}

All existing algorithms take their input data in the OI-FITS format
\citep{Pauls_et_al-2005-oifits}.  This format stores the optical
interferometric data in FITS binary tables.  More generally, a FITS file
\citep{Pence_et_al-2010-FITS} consists of a number of so-called header data
units (HDU), each HDU having an header part with scalar parameters
specified by FITS keywords\footnote{A FITS keyword consists of upper case
  latin letters, digits, hyphen or underscore characters and has at least
  one character and at most 8 characters.} and a data part.  The
header part obeys a specific format but it is textual and easy to read for
an human.  The data part usually consists of binary data stored in various
forms.  In this document we only consider FITS \emph{images} (that is a
multidimensional array of values of the same data type) and FITS
\emph{binary tables}.  These are sufficient for our purposes.

By adding FITS HDU or introducing new FITS keywords, it is possible to
provide any additional data (\emph{e.g.} the initial image) and
parameters required to run the image reconstruction process.  Thus, it
is proposed that the input file be a valid OI-FITS file (resulting
from the merging of all the optical interferometric data to process)
with the addition of HDU containing further binary data needed for the
reconstruction (for instance, the initial image) and with scalar input
settings provided as the values of FITS keywords.  This is detailed in
the following subsections.

Note that the FITS standard states that angular measurements expressed
as floating-point values and specified with reserved keywords
\emph{should} be given in degrees.

The HDUs that are specific to the interface specification described in this
document have \KEYWORD{EXTNAME} values prefixed with \STRING{IMAGE-OI}.
This is to distinguish them from OIFITS HDUs, which use the prefix
\STRING{OI\_}.


\subsection{Scalar input parameters}

In version 1 of OIFITS, the first (primary) HDU is almost empty.  However,
the draft version 2 of OIFITS introduces primary header keywords
summarizing the data and giving its provenance.  We propose, therefore,
storing the scalar image reconstruction input parameters in a separate HDU.
In anticipation of perhaps needing to store vector parameters in future,
this HDU shall be a binary table HDU.  This binary table shall have
\KEYWORD{EXTNAME} keyword set to \STRING{IMAGE-OI~INPUT~PARAM} and shall
contain all of the non-image parameters, with the exception of the pixel
size and the dimensions of the reconstructed image which are specified by
those of the initial image (see below), itself stored in a dedicated
(image) HDU. Table~\ref{tab:input-params} gives some examples of the input
parameters.


\subsection{Data selection}

To keep things simple, any sophisticated selection, merging or editing of
the data should be done by a separate tool.  The image reconstruction
software applications shall assume that they receive clean input data.
There are however a few parameters devoted to the selection of data.  The
\KEYWORD{TARGET} keyword specifies the name of the target object to
reconstruct.  The value of this keyword should match one of the identifiers
in the column \KEYWORD{TARGET} in the \texttt{OI\_TARGET} binary table of
the OIFITS file.  In order to restrict the types of interferometric data
used for the reconstruction, keywords \KEYWORD{USE\_VIS},
\KEYWORD{USE\_VIS2} and \KEYWORD{USE\_T3} should be set with values
specifying which complex visibility data, powerspectrum data and bispectrum
data to use if any.  More specifically, keywords \KEYWORD{USE\_VIS} and
\KEYWORD{USE\_T3} take string values which are \STRING{NONE}, \STRING{ALL},
\STRING{AMP} or \STRING{PHI} to indicate whether all, none, only the
amplitude or only the phase of such data have to be used.  Keyword
\KEYWORD{USE\_VIS2} has a boolean value indicating whether to consider the
powerspectrum data.  Not all algorithms can use all types of data and the
values of these keywords should be set (in the output file) so as to
reflect what was really used.

Whatever these settings, the image reconstruction software have to
honor the \KEYWORD{FLAG} column of the OI data.  We recall that this
field has a logical value which is true when the corresponding piece
of data should be discarded and false when it should be considered.

Wavelength selection is discussed in the next section.


\subsection{Wavelength range}

The primary objective is to consider monochromatic image reconstruction.  The
interferometric data are generally available at many wavelengths.  The result
will, in fact, be a gray image of the target built from the data in the
wavelength range (inclusive) specified by the FITS keywords \KEYWORD{WAVE\_MIN}
and \KEYWORD{WAVE\_MAX}. The wavelength range is given in meters.


\subsection{Initial image}

All reconstruction algorithms are iterative and require an initial
image to start with. The initial image is provided as a FITS image in
one of the HDU of the input file. The primary HDU can be used to store
the initial image (however \emph{cf.} the discussion about which image
to store in the primary HDU). The pixel size and the dimensions of the
initial image determine that of the reconstructed image.  The
dimensions are given by the FITS keywords \KEYWORD{NAXIS1} and
\KEYWORD{NAXIS2} while the pixel size is given by the FITS keywords
\KEYWORD{CDELT1} and \KEYWORD{CDELT2} (both values must be the same).

It is not intended that the image reconstruction algorithm be able to
deal with any possible world coordinate system (WCS) nor with any
possible coordinate units.  We therefore restrict to WCS which can be
fully specified with \KEYWORD{CRPIX\VARIABLE{i}},
\KEYWORD{CRVAL\VARIABLE{i}}, \KEYWORD{CDELT\VARIABLE{i}},
\KEYWORD{CTYPE\VARIABLE{i}}, and \KEYWORD{CUNIT\VARIABLE{i}} keywords
(where \VARIABLE{i} is the axis number). Other WCS keywords like
\KEYWORD{CROTAn}, \KEYWORD{PC\VARIABLE{i}\_\VARIABLE{j}},
\KEYWORD{DC\VARIABLE{i}\_\VARIABLE{j}}, should not be specified as
their default values (according to FITS standard) are suitable for us.
The same WCS conventions hold for any output image produced by the
reconstruction algorithm.  For any external software to correctly
display the images, the parameters of the WCS must be completely and
correctly specified.

The conventions are that the first image axis corresponds to the right
ascension (RA) and second image axis corresponds to the declination
(DEC) both relative to the center of the field of view (FOV) specified
by the keywords \KEYWORD{CRPIX\VARIABLE{i}} (in fractional pixel
units). If keywords \KEYWORD{CRPIX\VARIABLE{i}} are omitted, they
default to the geometric center of the FOV. The pixel size (which is
specified by the absolute values of \KEYWORD{CDELT1} and
\KEYWORD{CDELT2}) must be the same in both direction.  Following
standard conventions to display an image, to have the relative right
ascension (RA) oriented toward East to correspond to the left (first
columns) of the image, \KEYWORD{CDELT1} should be strictly negative,
while, to have the relative declination (DEC) oriented toward North to
correspond to the top (last rows) of the image,  \KEYWORD{CDELT2}
should be strictly positive. By default, the physical coordinate units
are in degrees; otherwise \KEYWORD{CUNIT\VARIABLE{i}} can be
\verb+'deg'+ for degrees or \verb+'arsec'+ for arcseconds.
Table~\ref{tab:wcs} summarizes these rules.

The OI-FITS norm specifies that \KEYWORD{RAEP0} and \KEYWORD{DECEP0}
in \KEYWORD{OI\_TARGET} are the coordinates of the phase center, which
we assume to be the absolute (not relative) world coordinates of the
reference pixel in the reconstructed image.  \oops{With these
conventions, the relative world coordinates of the FOV center
(specified by keywords \KEYWORD{CRVAL\VARIABLE{i}}) should be
$(0,0)$.}


\begin{table}
 \centering
  \caption{FITS keywords used to specify the Wold Coordinate System(WCS).
    All these keywords refer to a given axis denoted as \VARIABLE{i}.}
    \label{tab:wcs}
    
    \medskip
    
  \begin{tabular}{lll}
  \hline
  Keyword & Description & Default \\
  \hline
  \hline
  \KEYWORD{CRPIX\VARIABLE{i}} & Index index of the reference pixel &
  Geometric center \\
   & (start at 1 and can be fractional) \\
  \hline
  \KEYWORD{CRVAL\VARIABLE{i}} & Physical coordinate of ref. pixel & 0.0 \\
  \hline
  \KEYWORD{CDELT\VARIABLE{i}} & Physical coordinate increment & \\
  \hline
  \KEYWORD{CTYPE\VARIABLE{i}} & Coordinate name
  & \STRING{RA---TAN} for 1st axis,\\
  && \STRING{DEC--TAN} for 2nd axis \\
  \hline
  \KEYWORD{CUNIT\VARIABLE{i}} & Units of the physical coordinates&
  Degree \\
  \hline
  \end{tabular}
\end{table}

\oops{\texttt{fitsverify} gives a warning if \KEYWORD{CTYPE\slshape i} are
omitted. Should these given as \STRING{RA} and \STRING{DEC}, meaning linear
axes, or as \STRING{RA---SIN} and \STRING{DEC--SIN}, or as \STRING{RA---TAN}
and \STRING{DEC--TAN} which specify the appropriate non-linear transformations
between the tangent plane and the celestial sphere? Eric: In the document, I
have assumed that \STRING{RA---TAN} and \STRING{DEC--TAN} are the correct ones;
can we check whether this is true?}

%==============================================================================
\section{Output format}

It is proposed that the output format be as similar as possible as the input
format.  The output file must provide the reconstructed image but also some
information for analyzing the result.  In particular, as each imaging
algorithm may implement its own method for estimate the complex visibilities
given the image of the object, it is necessary that the model of every fitted
data point be computed by the algorithm itself rather than by another tool.

\subsection{Input parameters}

The output file shall contain a copy of the input parameters. These
shall be stored in the same way as in the input file, except that the
initial image is not stored in the primary HDU (this location is
reserved for the final/current image as explained below).

The image reconstruction software shall write out values for all of
the input parameters that it accepts, not just those that were
provided in the input file. Thus the output file will contain default
or automatically-chosen values for the remaining parameters.

\oops{Need to define conventions for EXTNAME and HDUNAME of all image HDUs}

\subsection{Final/current image}

To compare the initial and the final images, they must be stored in
different HDU, the FITS keyword \KEYWORD{HDUNAME} is used to
distinguish them (the EXTNAME keyword should not be used in the
primary HDU). As explained previously, the dimensions, pixel size and
orientation of the output image(s) are the same as the initial image.

%The value of the FITS keyword \KEYWORD{EXTNAME} is \STRING{INITIAL IMAGE} and
%\STRING{FINAL IMAGE} for the initial and final images respectively.

As most image viewers\footnote{not all softwares deal with FITS files}
are only capable of displaying the image stored in the primary HDU, we
suggest storing the initial image in the primary HDU of the input
file, but storing the final or current image in the primary HDU of the
output file.  The idea is to have the most relevant image stored in
the primary HDU.  For the image reconstruction algorithm and for any
software designed to display or analyze the results, the different
images are distinguished by their names (given by their
\KEYWORD{HDUNAME} keyword).  \oops{OK but the last image should have a
  specific name, perhaps set \KEYWORD{LAST\_IMG} keyword in the
  \emph{output} parameters HDU with name of the final image and let
  all reconstructed images be called:
\begin{quote}
\KEYWORD{HDUNAME = 'IMAGE-OI OUTPUT\VARIABLE{n}'}
\end{quote}
with \VARIABLE{n} the iteration number.}

In order to continue the iterations of a previous reconstruction run,
the image reconstruction algorithm may be started with the final image
instead of the initial one.  To that end, there must be some means to
specify the starting image for the reconstruction.  This is the
purpose of the \KEYWORD{INIT\_IMG} keyword (see
Table~\ref{tab:input-params}) in the input parameters HDU which
indicates the HDUNAME of the initial image. This HDUNAME must be
unique within the file.

\subsubsection{Output parameters}

Any scalar output parameters from the image reconstruction shall be
stored in a binary table HDU with \KEYWORD{EXTNAME} of
\STRING{IMAGE-OI OUTPUT PARAM}. Examples of scalar outputs include
$\chi^2$, any regularization parameters estimated from the data, and
any model parameters that are not image pixels (for example stellar
disk parameters in SPARCO).

\oops{Need to list standard output parameters in a table. Include
  something to identify the algorithm that was used?}

\subsection{Model of the data}

The OI-FITS format \citep{Pauls_et_al-2005-oifits} specifies that
optical interferometric data be stored in binary tables as columns
with specific names.  As there is no restriction that the tables only
contain the columns specified by the standard, we propose to store the
model of the data in the same tables by adding new columns. The
additional columns have the prefix \STRING{NS\_MODEL\_} to distinguish them from
the columns defined by the OIFITS standard. The names of the new
columns are listed in Table~\ref{tab:model-columns}.  In this way it
is very easy to compare the actual data and the corresponding model
values as computed from the reconstructed image and the instrumental
model assumed by the reconstruction algorithm.  Another advantage of
this convention is that the same format can be exploited to store the
values given by model fitting software.

\renewcommand{\ROW}[2]{\texttt{#1} & \texttt{D(\textsl{NWAVE})} & #2 \\}
\renewcommand{\ROWTITLE}[1]{\multicolumn{3}{c}{\textbf{#1}}\\}

\begin{table}
\caption{Colums inserted in OI-FITS binary tables to store the values given by
the model.  \texttt{\textsl{NWAVE}} is the number of wavelengths.
\label{tab:model-columns}}
\begin{tabular}{lcl}
\hline
\hline
\ROWTITLE{New columns in \texttt{OI\_VIS} table}
Label & Format & Description \\
\hline
\ROW{NS\_MODEL\_VISAMP}{Model of the visibility amplitude}
\ROW{NS\_MODEL\_VISAMPERR}{Model of the error in visibility amplitude \oops{(optional)}}
\ROW{NS\_MODEL\_VISPHI}{Model of the visibility phase in degrees}
\ROW{NS\_MODEL\_VISPHIERR}{Model of the error in visibility phase in degrees \oops{(optional)}}
\hline
\hline
\ROWTITLE{New column in \texttt{OI\_VIS2} table}
Label & Format & Description \\
\hline
\ROW{NS\_MODEL\_VIS2}{Model of the squared visibility}
\ROW{NS\_MODEL\_VIS2ERR}{Model of the error in squared visibility \oops{(optional)}}
\hline
\hline
\ROWTITLE{New columns in \texttt{OI\_T3} table}
Label & Format & Description \\
\hline
\ROW{NS\_MODEL\_T3AMP}{Model of the triple-product amplitude}
\ROW{NS\_MODEL\_T3AMPERR}{Model of the error in triple-product amplitude \oops{(optional)}}
\ROW{NS\_MODEL\_T3PHI}{Model of the triple-product phase in degrees}
\ROW{NS\_MODEL\_T3PHIERR}{Model of the error in triple-product phase in degrees \oops{(optional)}}
\hline
\end{tabular}
\end{table}

%==============================================================================
\section{Command line tools}

It would be useful to provide command line tools for creating and
editing input files (e.g. to change a parameter), in addition to the
graphical user interface. Such tools would be useful for scripting
tests, for example.

Some example command lines are shown below. The first creates a new input
file named \verb+myinput.fits+ and sets the initial image from the
primary HDU of \verb+myimage.fits+. The second command changes
parameter values in an existing file.

\begin{verbatim}
oi-image create myinput.fits CDELT1=0.25 MAXITER=200 INIT_IMG=myimage.fits
oi-image edit myinput.fits RGL_ALPH=1e-3 USE_T3A=NONE
\end{verbatim}

%==============================================================================
\appendix
\section{Glossary}

\begin{description}
\item[DEC] Declination;
\item[FITS] Flexible Image Transport System;
\item[FOV] Field of view;
\item[GUI] Graphical User Interface;
\item[HDU] Header Data Unit;
\item[OI-FITS] Optical Interferometric exchange data format;
\item[RA] Right Ascension;
\item[WCS] World Coordinate System;
\end{description}

%==============================================================================
\bibliographystyle{plainnat}
\bibliography{OI-Interface}

\end{document}

